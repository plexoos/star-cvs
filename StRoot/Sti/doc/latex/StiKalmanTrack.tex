\documentclass{revtex4}
\usepackage{graphicx}

\newcommand{ \be }{\begin{equation}}
\newcommand{ \ee }{\end{equation}}
\newcommand{ \bea }{\begin{eqnarray}}
\newcommand{ \eea }{\end{eqnarray}}
\newcommand{ \la }{\langle}
\newcommand{ \ra }{\rangle}

\begin{document}

\section{Track Length Determination}

The class {\em StiKalmanTrack} implements the method 
{\em double getTrackLength() const} defined in the 
abstract interface {\em StiTrack} to calculate and return 
the track length. 

The track length is defined as the path length between the 
outer and inner (first and last) points assigned on a track.

It is assumed for practical purposes that the track does not 
loop inside the detector. The track determination proceeds as follows:

\begin{itemize}
\item Obtain the positions of the first ($\vec{P}_1$) and 
last ($\vec{P}_2$) points on the track. 
\item Determine the longitudinal and transverse distance, 
$\Delta z=\vec{P}_{1,z} - \vec{P}_{2,z}$,
and $d=\sqrt{(\vec{P}_{1,x} - \vec{P}_{2,x})^2+
(\vec{P}_{1,y} - \vec{P}_{2,y})^2}$
\item Determine the transverse path length, $s$, based on the 
curvature, $C=1/R$, of the track at the inner point with 
the following expression:

\be
s = 2 R {\rm arcsin}(Cd/2)
\ee

\item Determine and return the full length $L=\sqrt{\Delta z^2+s^2}$.
\end{itemize}

\end{document}